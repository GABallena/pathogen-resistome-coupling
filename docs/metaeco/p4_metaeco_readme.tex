% Portfolio-safe copy: identifiers generalized; no data included.
\documentclass[11pt]{article}
\usepackage[a4paper,margin=1in]{geometry}
\usepackage{amsmath,amssymb,mathtools}
\usepackage{booktabs}
\usepackage{hyperref}
\usepackage{enumitem}
\usepackage{graphicx}
\usepackage{siunitx}
\usepackage{courier}
\usepackage{listings}
\lstset{basicstyle=\ttfamily\small,breaklines=true}

\title{PROJECT Ecology Analytics: Background, Equations, and Practical Workflows}
\author{PROJECT Ecology Toolkit}
\date{\today}

\begin{document}
\maketitle
\tableofcontents

\section{What this pipeline does (in one paragraph)}
Given a merged abundance table from two profiling tools (e.g., \texttt{metaphlan}, \texttt{bracken}) and optionally per-sample read depths \((N)\), this pipeline (i) compares tools in a compositional space (CLR/ALR/PLR) and reports Pearson/Spearman plus the Aitchison distance, (ii) estimates unbiased factorial moments to recover mean proportions \(\bar{x}_i\) for each taxon, (iii) fits abundance fluctuation distributions (AFDs) using a Negative Binomial (Gamma–Poisson) model and optionally Poisson–lognormal, (iv) performs goodness-of-fit and posterior predictive checks, and (v) derives detection probabilities, \(N^{*}(q)\) sequencing-depth targets, and richness forecasts \(R(N)\) with slopes \(R'(N)\). Utilities include metadata-driven sample selection, negatives/decontam, stratified diagnostics, Taylor’s law, and MAD lognormal checks.

\section{Compositional background}
Let a sample be a composition \(\mathbf{x}=(x_1,\dots,x_D)\) with \(x_i>0\) and \(\sum_i x_i=1\).

\paragraph{Centered log-ratio (CLR).}
\[
\mathrm{clr}(x_i) \;=\; \log x_i - \frac{1}{D}\sum_{j=1}^D \log x_j.
\]
Distances in CLR space correspond to Aitchison geometry on the simplex. In practice, zeros require replacement.

\paragraph{Additive log-ratio (ALR).}
Choose a reference part \(r\) (or an average of a set of references) and set
\[
\mathrm{alr}_r(x_i) \;=\; \log x_i - \log x_r.
\]
Robust if the reference is stable.

\paragraph{Pairwise log-ratios (PLR).}
For taxa \(i,j\) present in both tools:
\[
\log\frac{x_i}{x_j} \;=\; \log x_i - \log x_j.
\]
Operates on pairs; robust to sparsity; ignores taxa not present in both vectors.

\paragraph{Aitchison distance.} For two compositions (after a common subcomposition/zero handling),
\[
d_A(\mathbf{x},\mathbf{y}) \;=\; \left\| \mathrm{clr}(\mathbf{x}) - \mathrm{clr}(\mathbf{y}) \right\|_2.
\]
The pipeline reports this as \texttt{aitchison} (\emph{Euclidean norm of CLR differences} over the per-sample common subcomposition).

\section{AFDs: Gamma--Poisson (Negative Binomial) and alternatives}
We model per-sample counts \(X_{is}\) for taxon \(i\) in sample \(s\) with depth \(N_s\). Let \(\bar{x}_i\) be the mean proportion across samples (estimated from unbiased factorial moments). The mean count is
\[
\mu_{is} \;=\; \bar{x}_i\,N_s.
\]
\subsection{Negative Binomial parameterization}
We use the mean--size parameterization:
\[
X \sim \mathrm{NB}(\mu,\beta), \qquad
\mathbb{E}[X]=\mu,\quad
\mathrm{Var}(X)=\mu+\frac{\mu^2}{\beta}.
\]
The pmf:
\[
\Pr(X=n\mid \mu,\beta) \;=\;
\frac{\Gamma(n+\beta)}{\Gamma(\beta)\,n!}\,
\left(\frac{\beta}{\beta+\mu}\right)^{\beta}
\left(\frac{\mu}{\beta+\mu}\right)^n.
\]
Zero probability and its Poisson limit (\(\beta\to\infty\)):
\[
p_0(\mu,\beta) = \left(\frac{\beta}{\beta+\mu}\right)^{\beta}
\;\xrightarrow[\beta\to\infty]{}\; e^{-\mu}.
\]

\subsection{Detection probability, occupancy-as-detection}
Given \(\mu_{is}=\bar{x}_i N_s\):
\[
\Pr(X_{is}>0) \;=\; 1 - p_0(\mu_{is},\beta_i).
\]
Empirical ``occupancy'' across samples is the mean detection probability:
\[
\widehat{\psi}_i \;=\; \frac{1}{S}\sum_{s=1}^S \bigl(1-p_0(\bar{x}_i N_s,\beta_i)\bigr).
\]
(Here this is a \emph{detection} notion, not a hierarchical occupancy model.)

\subsection{Depth for target detection \texorpdfstring{$N^{*}(q)$}{N*(q)}}
Solve \(\Pr(X>0) \ge q\) for \(N\). With \(\mu=\bar{x}N\):
\[
N^{*}(q) \;=\;
\begin{cases}
-\dfrac{\log(1-q)}{\bar{x}}, & \beta=\infty \text{ (Poisson)},\\[1em]
\dfrac{\beta}{\bar{x}}\left[(1-q)^{-1/\beta}-1\right], & \beta<\infty.
\end{cases}
\]

\subsection{Richness and slope vs depth}
Let \(p_{0,i}(N)=p_0(\bar{x}_i N, \beta_i)\). Then
\[
R(N) \;=\; \sum_i \bigl(1-p_{0,i}(N)\bigr),
\qquad
R'(N) \;=\; \sum_i \frac{\partial}{\partial N}\bigl(1-p_{0,i}(N)\bigr).
\]
Closed forms:
\[
R'(N) =
\begin{cases}
\sum_i \bar{x}_i\,e^{-\bar{x}_i N}, & \beta_i=\infty,\\[0.5em]
\sum_i \bar{x}_i\left(1+\dfrac{\bar{x}_i N}{\beta_i}\right)^{-(\beta_i+1)}, & \beta_i<\infty.
\end{cases}
\]

\section{Unbiased factorial-moment estimators}
Given counts \(n_{is}\) with depths \(N_s\), define totals
\[
D_1=\sum_s N_s,\quad
D_2=\sum_s N_s(N_s-1),\quad
D_3=\sum_s N_s(N_s-1)(N_s-2).
\]
Then for each taxon \(i\),
\[
\hat{\mu}_1=\frac{\sum_s n_{is}}{D_1},\quad
\hat{\mu}_2=\frac{\sum_s n_{is}(n_{is}-1)}{D_2},\quad
\hat{\mu}_3=\frac{\sum_s n_{is}(n_{is}-1)(n_{is}-2)}{D_3}.
\]
Finite-depth ``naive'' moments (on proportions \(n_{is}/N_s\)) have curvature away from the factorial moments; the code predicts that curvature and reports residuals with bootstrap CIs.

\section{Goodness-of-fit and diagnostics}
\paragraph{$\chi^2$ GOF per taxon.} Bin counts and compare observed vs NB-expected frequencies (merged bins ensure expected \(\ge 5\)); adjust df for fitted parameters; BH-FDR \(q\)-values are reported.

\paragraph{Posterior predictive (PPC).} Simulate zero fractions under fitted parameters; two-sided tail probability for observed zero fraction.

\paragraph{Taylor’s law.} Regress \(\log\mathrm{Var}\) vs \(\log\mathrm{Mean}\) across taxa; test \(H_0:\) slope \(=2\).

\paragraph{MAD lognormal.} Anderson–Darling test for normality of \(\log \hat{\mu}_1\) (Shapiro--Wilk when \(n\le 5000\)).

\section{Agreement metrics between tools}
For each sample, transform both abundance vectors with chosen transform, then report:
\begin{itemize}[leftmargin=1.1em]
\item Pearson correlation (linear) and Spearman correlation (rank).
\item \texttt{aitchison}: Euclidean norm of CLR differences on that sample’s common subcomposition.
\end{itemize}
Tip: with many zeros, PLR or ALR with a top-\(k\) pivot is more stable than CLR.

\section{Memory \& numerics (practical notes)}
\begin{itemize}[leftmargin=1.1em]
\item Zero probabilities use IEEE-safe exponent bounds (around \(\approx 745\) for double precision) to avoid overflow/underflow.
\item Presence matrices are avoided for large data; per-fold occupancies are computed by grouping where possible.
\item Block bootstraps (\texttt{--bootstrap-block-col}) respect clustered designs (subjects/sites/time).
\end{itemize}

\section{Workflow recipes (do-this-to-answer-that)}
Below, \emph{replace} file names and flags to your context. All commands assume Python entrypoint \texttt{p4\_eco.py}.

\subsection{Am I sequencing deep enough? What depth hits 95\% detection for key taxa?}
\begin{lstlisting}
python p4_eco.py \
  --merged merged.tsv --counts counts.tsv \
  --tool-pair metaphlan,bracken \
  --afd nb --target-detect 0.95 \
  --out-dir results/depth
\end{lstlisting}
Read \texttt{depth\_sufficiency\_per\_species.tsv}:
\begin{itemize}[leftmargin=1.1em]
\item \texttt{N\_star\_q}: the \(N^{*}(q)\) per taxon.
\item \texttt{is\_sufficient\_at\_current}: at median observed \(N\), do you meet the target?
\end{itemize}
  	extbf{If many taxa are insufficient}, consider higher depth or focus on taxa with higher \(\bar{x}_i\) or lower overdispersion (larger \(\beta_i\)).

\subsection{How many more species do I gain if I increase depth by \texorpdfstring{$X$}{X}?}
\begin{lstlisting}
python p4_eco.py \
  --merged merged.tsv --counts counts.tsv \
  --afd nb --depths 5e6,1e7,2e7,4e7 \
  --out-dir results/richness
\end{lstlisting}
Inspect \texttt{richness\_vs\_depth.tsv}:
\begin{itemize}[leftmargin=1.1em]
\item \texttt{R}: expected richness at each depth.
\item \texttt{R\_prime}: marginal gain (\(dR/dN\)); when this is near zero, you’re in diminishing returns.
\end{itemize}

\subsection{Does NB fit my counts? Should I prefer Poisson--lognormal for some taxa?}
\begin{lstlisting}
python p4_eco.py \
  --merged merged.tsv --counts counts.tsv \
  --afd both --gof-alpha 0.05 --ppc-draws 1000 \
  --out-dir results/modelcheck
\end{lstlisting}
\textbf{Look at} \texttt{afd\_fit.tsv}:
\begin{itemize}[leftmargin=1.1em]
\item \texttt{gof\_q} (BH-FDR): small values indicate lack of fit.
\item \texttt{afd\_winner} (AIC): NB vs PLN preference per taxon.
\item \texttt{ppc\_zero\_p}: PPC on zero fraction (extreme tails suggest misfit).
\end{itemize}
If many failures: enable \texttt{--fit-per-species-beta}, check denominators, or accept PLN for those taxa.

\subsection{Which correlation (Pearson vs Spearman) should I use at a planned depth?}
\begin{lstlisting}
python p4_eco.py \
  --merged merged.tsv --counts counts.tsv \
  --forecast-depths 5e6,1e7,2e7 \
  --out-dir results/forecast
\end{lstlisting}
See \texttt{agreement\_forecast\_vs\_depth.tsv} and the summary:
when predicted zero-fraction is low, Pearson is suggested; else Spearman.

\subsection{I have lots of zeros. How do I compare tools robustly?}
\begin{lstlisting}
python p4_eco.py \
  --merged merged.tsv --tool-pair metaphlan,bracken \
  --agreement-mode alr --alr-topk 3 \
  --out-dir results/agreement_alr
\end{lstlisting}
Or use PLR:
\begin{lstlisting}
python p4_eco.py \
  --merged merged.tsv --tool-pair metaphlan,bracken \
  --agreement-mode plr --plr-maxpairs 50000 \
  --out-dir results/agreement_plr
\end{lstlisting}

\subsection{Decontam with negatives (prevalence-based is safer first)}
\begin{lstlisting}
python p4_eco.py \
  --merged merged.tsv --counts counts.tsv \
  --metadata meta.xlsx --sample-col SAMPLE_CODE \
  --decontam prevalence --prevalence-thresh 0.5 \
  --out-dir results/decontam_prev
\end{lstlisting}
If you must subtract:
\begin{lstlisting}
python p4_eco.py \
  --merged merged.tsv --counts counts.tsv \
  --metadata meta.xlsx --sample-col SAMPLE_CODE \
  --decontam subtract --out-dir results/decontam_sub
\end{lstlisting}
Check \texttt{negatives\_background.tsv} and \texttt{decontam\_dropped\_taxa.tsv}.

\subsection{Clustered study (subjects/time/sites): honest uncertainty}
\begin{lstlisting}
python p4_eco.py \
  --merged merged.tsv --counts counts.tsv \
  --metadata meta.xlsx --sample-col SAMPLE_CODE \
  --bootstrap-block-col subject_id \
  --cv-folds 5 --cv-metric occupancy_mse \
  --out-dir results/clustered
\end{lstlisting}
This widens CIs (good!) and makes occupancy calibration out-of-sample.

\section{Flag guide (what they do \& how tweaking changes behavior)}
\subsection*{Compositional transforms}
\begin{itemize}[leftmargin=1.1em]
\item \texttt{--agreement-mode \{clr,alr,plr\}}:
CLR is powerful with few zeros; ALR stabilizes via a pivot; PLR is robust for sparse data.
\item \texttt{--pseudocount} (CLR/ALR only): larger values damp zero influence; smaller values emphasize rare parts (risking noise).
\item \texttt{--alr-ref}, \texttt{--alr-topk}: pick a stable biological reference or average of top-\(k\) abundant taxa.
\item \texttt{--plr-maxpairs}: cap pairs for speed; larger caps reduce variance of the correlations.
\end{itemize}

\subsection*{AFD fitting \& model checking}
\begin{itemize}[leftmargin=1.1em]
\item \texttt{--afd \{nb,pln,both\}}: choose NB, PLN, or compare by AIC.
\item \texttt{--fit-per-species-beta}: per-taxon \(\beta_i\); needs adequate data per taxon; improves fit heterogeneity.
\item \texttt{--beta-grid}: robust grid search for shared \(\beta\).
\item \texttt{--cv-folds}, \texttt{--cv-metric}: k-fold calibration for occupancy/zero fraction; higher \(k\) = more compute, less optimistic.
\item \texttt{--ppc-draws}: more draws = stabler PPC, higher runtime.
\item \texttt{--pln-quadrature}: Hermite nodes; higher = more accurate PLN likelihood, slower.
\end{itemize}

\subsection*{Detection planning \& forecasting}
\begin{itemize}[leftmargin=1.1em]
\item \texttt{--target-detect} \(q\): fraction for \(N^{*}(q)\); larger \(q\Rightarrow\) larger \(N^{*}\).
\item \texttt{--depths}: compute \(R(N),R'(N)\) at these \(N\).
\item \texttt{--forecast-depths}: predict zero fractions and suggest Pearson vs Spearman per depth.
\end{itemize}

\subsection*{Selection, metadata, negatives}
\begin{itemize}[leftmargin=1.1em]
\item \texttt{--include-regex}: quick subsetting by sample IDs.
\item \texttt{--metadata}, \texttt{--sample-col}, \texttt{--cohort-col}, \texttt{--cohort-allow}, \texttt{--date-col}, \texttt{--exclude-old-before}: inclusion/exclusion and negatives tagging.
\item \texttt{--neg-regex}: fallback negatives detection by ID patterns.
\item \texttt{--decontam \{none,subtract,prevalence\}}, \texttt{--prevalence-thresh}: recommended to start with prevalence; subtraction can over-correct at low \(N\).
\item \texttt{--min-occupancy}: model-guided filtering threshold before re-agreement (higher threshold \(\Rightarrow\) fewer taxa, often higher agreement).
\end{itemize}

\subsection*{Uncertainty \& stats}
\begin{itemize}[leftmargin=1.1em]
\item \texttt{--n-bootstrap}: CIs for curvature residuals.
\item \texttt{--bootstrap-block-col}, \texttt{--bootstrap-max}, \texttt{--bootstrap-eps}: clustered resampling and early stopping.
\item \texttt{--gof-alpha}, \texttt{--report-holm}: GOF threshold and conservative multiplicity option.
\item \texttt{--stratify}, \texttt{--batch-col}: per-batch Taylor/MAD reporting.
\end{itemize}

\section{Interpreting \& validating}
\paragraph{Agreement outputs.} If \texttt{spearman} $\gg$ \texttt{pearson}, relationship is monotonic but nonlinear; zeros likely dominate.

\paragraph{AFD outputs.} Tiny \(\beta_i\) indicate heavy overdispersion; cross-check GOF and PPC. Use \(N^{*}(q)\) and \(R(N)\) to plan depth. 

\paragraph{Validation with ground truths.}
\begin{itemize}[leftmargin=1.1em]
\item \textbf{Mock communities}: compare predicted vs observed \(N^{*}(q)\); check if \(R(N)\) tracks known richness as depth increases.
\item \textbf{Spike-ins}: verify \(\bar{x}_i\) and \(\beta_i\) recovery; PPC should not systematically reject spikes.
\item \textbf{Train/test splits}: use \texttt{--cv-folds} and compare predictive zero fractions to held-out.
\end{itemize}

\section{Caveats \& extensions}
\begin{itemize}[leftmargin=1.1em]
\item ``Occupancy'' here is detection; full occupancy--detection hierarchies are future work.
\item Compositional constraints persist; for across-sample comparability in agreement metrics, pre-filter to a fixed subcomposition (e.g., taxa present in \(\ge\)X\% samples).
\item Phylogeny, ecological nulls, and hierarchical batch models are not yet integrated.
\end{itemize}

\bigskip
\noindent\textbf{Reproducibility tips:} pin \texttt{--seed}, version input TSVs and metadata, keep \texttt{summary.txt} with the exact command used.

\end{document}
